\documentclass[conference]{IEEEtran}
\usepackage{cite}
\usepackage{amsmath,amssymb,amsfonts}
\usepackage{algorithmic}
\usepackage{graphicx}
\usepackage{textcomp}
\usepackage{xcolor}
\usepackage{enumerate}
\def\BibTeX{{\rm B\kern-.05em{\sc i\kern-.025em b}\kern-.08em
    T\kern-.1667em\lower.7ex\hbox{E}\kern-.125emX}}
\begin{document}

\title{The Influence of Taste on Music Perception and Preference\\
{\large End Semester Exam - Music, Mind \& Technology}
}

\author{\IEEEauthorblockN{ Sharma, Aaryan.}
\IEEEauthorblockA{\textit{Centre for Security, Theory, and Algorithmic Research} \\
\textit{IIIT, Hyderabad }\\
Vadodara, India \\
aaryan.s@research.iiit.ac.in}
}

\maketitle

\begin{abstract}
This study aims to investigate the effect of taste on music perception and preference. A diverse sample of $50$ adults will be recruited to participate in the study, and four different taste stimuli (sweet, salty, bitter, and sour) will be presented to participants in the form of flavored solutions. A selection of 16 musical excerpts representing various genres, tempos, and emotional valences will be used. Participants will be asked to rate their enjoyment of the music on a 7-point Likert scale after tasting each solution. Data analysis will involve the use of repeated-measures ANOVA, post-hoc tests, correlation analyses, and multiple regression analysis. The findings of this study have the potential to inform the design of immersive dining and entertainment experiences, as well as contribute to the growing body of research on crossmodal correspondences.
\end{abstract}

\begin{IEEEkeywords}
Taste, Music, Perception, Crossmodal correspondences, Multisensory experience
\end{IEEEkeywords}

\section{Introduction}

Music is a universal language that transcends cultural and geographical boundaries, playing a significant role in human culture and experience (Levitin, 2007) \cite{b1}. It has the power to evoke emotions, memories, and even influence our taste preferences (Spence, 2021) \cite{b2}. The relationship between taste and music has been a topic of interest for researchers, musicians, and psychologists alike, as they seek to understand the complex interplay between sensory modalities and how they shape our perception and enjoyment of music (Crisinel \& Spence, 2011) \cite{b3}.

Previous research has demonstrated that music can influence the perception of taste, with certain musical parameters, such as pitch and tempo, affecting the perceived sweetness, bitterness, or sourness of food and beverages (Knoeferle et al., 2015) \cite{b4}. This phenomenon, known as cross-modal correspondence, suggests that our sensory experiences are not isolated but rather interconnected, with one modality potentially influencing the perception of another (Spence, 2011) \cite{b5}.

Building on this foundation, our study aims to investigate the effect of taste on music perception and preference, exploring how different taste sensations may alter the way we perceive and enjoy music. This research question is particularly relevant in the context of the growing interest in multisensory experiences, such as music-themed dining events and immersive concerts, where taste and music are intentionally combined to create unique and memorable experiences (Deroy et al., 2014) \cite{b6}.

To address this research question, we will examine the influence of various taste stimuli, representing the basic taste qualities of sweet, sour, salty, bitter, and umami, on participants' perception and preference of musical excerpts from different genres and emotional valences. By doing so, we aim to uncover potential cross-modal correspondences between taste and music perception, as well as to explore the role of individual differences, such as personality traits and cognitive styles, in shaping these relationships (Rentfrow \& Gosling, 2003) \cite{b7}.

This study contributes to the growing body of literature on cross-modal perception and multisensory experiences, providing valuable insights into the complex relationship between taste and music perception and preference. The findings of this study have potential implications for various fields, such as marketing, product design, and the entertainment industry, where understanding the interplay between sensory modalities can inform the creation of more engaging and immersive experiences (Spence, 2021) \cite{b2}.

The investigation of the effect of taste on music perception and preference offers a unique opportunity to explore the intricate connections between our sensory experiences and their impact on our enjoyment of music. By examining these relationships, we hope to deepen our understanding of the human experience of music and contribute to the development of innovative multisensory experiences that enrich our lives.

\section{Literature Review}
Knöferle and Spence (2012) conducted a series of experiments to investigate crossmodal correspondences between sounds and tastes \cite{b8}. Their key findings revealed that participants associated specific auditory features, such as pitch and timbre, with particular tastes. For example, sweet tastes were linked to high-pitched sounds, while bitter tastes were associated with low-pitched sounds. This study highlighted the existence of systematic mappings between auditory and gustatory modalities.

Crisinel and Spence (2013) examined the impact of pleasantness ratings on crossmodal associations between taste and pitch \cite{b9}. They found that participants consistently matched sweet and sour tastes with high-pitched sounds, while bitter and salty tastes were associated with low-pitched sounds. Interestingly, the strength of these associations was influenced by the pleasantness of the taste, with more pleasant tastes eliciting stronger crossmodal correspondences.

Wang and Spence (2015) assessed the effect of musical congruency on wine tasting in a live performance setting \cite{b10}. They discovered that congruent music-taste pairings, such as a sweet-sounding piano piece paired with a sweet wine, enhanced the overall experience of wine tasting. This study demonstrated the potential applications of crossmodal correspondences in real-world settings, such as enhancing the enjoyment of food and beverages through congruent music pairings.

The aim of our study differs from previous research in several ways. First, we aim to investigate the influence of taste on music perception and preference, rather than focusing solely on crossmodal correspondences between specific auditory and gustatory features. Second, our study will explore the effects of taste on music enjoyment across a broader range of musical parameters, including genre, tempo, and emotional valence. Also, by employing a larger and more diverse sample of participants, our study seeks to provide a more comprehensive understanding of the complex relationship between taste and music perception, taking into account individual differences and potential demographic influences.

\section{Methodology}
The null hypothesis ($H_0$) of the experiment is 
$H_0 = $ ``Taste does not have an effect on music perception and preference" while the alternate hypothesis ($H_1$) is 
$H_1 = $ ``Taste influences music perception and preference". The taste stimuli selected for the experiment are sweet, salty, bitter, and sour, which are commonly used in taste research.
The independent variable in the study is the taste stimuli provided to the participants, while the dependent variable is music enjoyment, measured using a 7-point Likert scale. Extraneous variables, such as time of day, external sounds, and temperature, should be controlled as much as possible. The study will employ a single experimental group, with each participant exposed to all four taste stimuli in a randomized order.
The musical excerpts in the study can be selected so as to represent a range of genres, tempos, and emotional valences. Participants will be asked to rate their enjoyment of the music on a 7-point Likert scale after tasting each solution. The platform for the experiment can be any online survey tool, which presents the taste stimuli and musical excerpts to the participants. The survey tool will record the participants' responses and stores them in a csv file. Participants can be recruited by circulating a Google form, and a total of $50$ adults should be included in the study. The study should be conducted in a controlled environment, and participants should be given instructions on how to taste the solutions and how to rate the music.
By employing a within-subject experimental design and controlling for extraneous variables, this study would provide a comprehensive examination of the relationship between taste and music perception and preference. The use of multiple taste stimuli and musical excerpts would add to the validity and reliability of the findings.
\subsection{Participants}
A diverse sample of adults ($N=50$) will be recruited for this study, ensuring a wide range of age, gender, and cultural backgrounds. Participants will be screened for any hearing impairments, taste disorders, or allergies to the taste stimuli used in the study.

\subsection{Materials}
Four different taste stimuli (sweet, salty, bitter, and sour) will be presented to participants in the form of flavored solutions, prepared using sucrose, sodium chloride, quinine, and citric acid, respectively. A selection of 16 musical excerpts representing various genres (classical, jazz, pop, and electronic), tempos (slow, medium, fast), and emotional valences (positive, negative) will be used. High-quality headphones and a computer-based interface for stimulus presentation and response collection will be employed.
\subsection{Procedure}

Participants will be seated in a quiet, comfortable room and provided with headphones. They will be given a brief overview of the study and instructions on how to rate their enjoyment of the music. Participants will be asked to rinse their mouths with water before tasting each solution to minimize carryover effects. They will then taste each of the four solutions while listening to the musical excerpts in a randomized order to avoid order bias. After each taste-music pairing, participants will rate their enjoyment of the music on a 7-point Likert scale, ranging from 1 (not at all enjoyable) to 7 (extremely enjoyable). The entire procedure will take approximately 45 minutes to complete.

\section{Results}
Data analysis will involve several statistical tests to examine the effects of taste on music preference and ensure the validity of the findings. The Shapiro-Wilk test will be used to assess the normality of the data. If the data is normally distributed, repeated-measures ANOVA will be conducted, with taste (sweet, salty, bitter, sour) and music (genre, tempo, valence) as within-subject factors. If the data is not normally distributed, non-parametric alternatives, such as the Friedman test, will be employed.

Post-hoc tests, such as Tukey's HSD for parametric data or Dunn's test for non-parametric data, will be conducted to identify specific taste-music pairings that significantly influence music enjoyment. Additionally, correlation analyses, using Pearson's correlation for parametric data or Spearman's rank correlation for non-parametric data, will be performed to explore potential relationships between taste preferences and musical preferences, as well as any demographic factors that may influence these associations.

Furthermore, multiple regression analysis will be conducted to determine the extent to which taste sensations can predict music enjoyment, while controlling for potential confounding variables, such as age, gender, and cultural background.

By employing a comprehensive set of statistical tests, this study aims to provide a robust and detailed examination of the relationship between taste and music perception, ensuring the validity and reliability of the findings.

\section{Conclusion}
This study aimed to provide a comprehensive understanding of the complex relationship between taste and music perception. By examining the influence of various taste sensations on music enjoyment across different genres, tempos, and emotional valences, we can gain valuable insights into the multisensory experience of music listening. The findings of this study have the potential to inform the design of immersive dining and entertainment experiences, as well as contribute to the growing body of research on crossmodal correspondences. Furthermore, understanding the role of taste in music perception may have implications for music therapy, gastronomy, and marketing, opening up new avenues for interdisciplinary research and collaboration.

\section{Limitations \& Future Scope}

Despite the valuable insights gained from this study, there are several limitations that should be considered when interpreting the results. One of the primary limitations is the use of a single measure of music enjoyment, namely the 7-point Likert scale. While this measure is widely used in music research, it may not capture the full range of emotional and cognitive responses to music. Future studies could consider using additional measures, such as physiological responses (e.g., heart rate variability) or self-report measures of emotional valence (e.g., the Positive and Negative Affect Schedule).

Another potential limitation of the study is the use of a limited range of taste stimuli. While the four taste qualities used in this study (sweet, salty, bitter, and sour) are commonly used in taste research, they may not fully capture the complexity of taste perception. Future studies could consider using a wider range of taste stimuli, such as umami or fatty tastes, to explore their influence on music perception and preference.
Additionally, the use of a single exposure to each taste-music pairing may not fully capture the long-term effects of taste on music enjoyment. Future studies could consider using a repeated exposure paradigm to examine how taste preferences may change over time and how this may influence music enjoyment.

The use of a relatively small and homogenous sample of participants may also limit the generalizability of the findings. While efforts can be made to recruit a diverse sample of participants, future studies could consider using larger and more diverse samples to ensure the validity and reliability of the findings.

There is also the possibility of demand characteristics creeping in if we inform the participants beforehand the hypothesis we are trying to test, though this can easily be eliminated by keeping the aim confidential, and only revealing the objective of the experiment after the experiment is conducted.

Despite these limitations, this study provides a valuable contribution to the growing body of research on the relationship between taste and music perception. Future studies could build on these findings by exploring the underlying mechanisms of taste-music interactions, such as the role of cognitive and emotional processes in shaping music enjoyment.

Additionally, future studies could consider the influence of individual differences, such as age, gender, and cultural background, on taste-music interactions. For example, previous research has shown that cultural background can influence crossmodal correspondences between sounds and tastes (Knöferle \& Spence, 2012), and it is possible that similar cultural differences may exist in taste-music interactions.

Exploring the potential applications of taste-music interactions in real-world settings, such as in the design of immersive dining and entertainment experiences can also be included in the scope of the future studies. By understanding the complex relationship between taste and music perception, we may be able to create more engaging and enjoyable experiences for consumers, leading to increased satisfaction and loyalty.

\section{Novice v/s Virtuoso}

This section is dedicated to discuss how a person who has not taken this course might approach this problem, and how the teachings from the course have helped us develop a more comprehensive and nuanced approach to the same problem.

Without the knowledge gained from the course, a person approaching the study of the influence of taste on music perception and preference might focus primarily on the auditory aspects of music, such as pitch, tempo, and timbre. They might overlook the potential influence of taste on music perception and the role of individual differences in shaping music preferences. Consequently, their study design might be limited in scope and depth, potentially leading to incomplete or inaccurate conclusions.

In contrast, the teachings from the course have provided us with a more comprehensive understanding of the complex relationship between music and human cognition, emotion, and perception. This knowledge has been invaluable in designing and conducting our study on the influence of taste on music perception and preference. Some of the key insights from the course that have informed our study include:
\begin{enumerate}[I.]
  \item Cross-modal Perception: The course introduced us to the concept of cross-modal perception, which explores the interplay between different sensory modalities and how they can influence our experience of music. This knowledge has informed the selection of taste stimuli for the study and has allowed for a more detailed examination of the influence of taste on music perception.
  \item Individual Differences: The course provided insights into the role of individual differences in shaping music preferences, such as the influence of personality traits, empathy, and cognitive styles. This knowledge has allowed for a more nuanced analysis of the data, taking into account the potential influence of individual differences on the relationship between taste and music perception.
  \item Relevance of Context: The course emphasized the importance of context in shaping music perception and preference. This understanding has informed the selection of musical excerpts for the study and has allowed for a more detailed examination of the influence of taste on music perception across a range of musical genres and emotional valences.

  \item Psychoacoustics and Audio Signal Processing: The course provided a foundation in psychoacoustics and audio signal processing, allowing for a more detailed analysis of the acoustic features of the musical excerpts used in the study. This knowledge has informed the selection of musical excerpts and has allowed for a more detailed examination of the influence of taste on music perception across a range of musical parameters.
\end{enumerate}
By applying these insights from the course, we have been able to design and conduct a study that provides a more comprehensive and nuanced understanding of the influence of taste on music perception and preference. The course has equipped us with the necessary tools and knowledge to approach the problem from multiple perspectives, allowing for a more detailed and insightful analysis of the data. In doing so, we have been able to surpass the limitations that a person without this knowledge might encounter, ultimately leading to a more robust and meaningful study.

\section{Course Summary}
The ``Music, Mind \& Technology" course provided a comprehensive exploration of the complex relationship between music and human cognition, emotion, and perception. The course covered a wide range of topics, from the evolutionary origins of music to the latest developments in neuroscience and digital signal processing. The key take-aways from the course can be summarized as follows:
\begin{enumerate}[I.]

\item Evolutionary Origins of Music: The course delved into the deep evolutionary roots of music, exploring theories that suggest music has served various functions throughout human history. These functions include social bonding, emotional expression, and cognitive enhancement. By understanding the evolutionary origins of music, we can better appreciate its significance in human culture and its impact on our cognitive and emotional experiences.

\item Music Preferences \& Individual Differences: The course emphasized the importance of individual differences in shaping music preferences and experiences. Factors such as personality traits, empathy, and cognitive styles play a significant role in determining how individuals perceive and respond to music. This understanding can help us develop personalized music recommendations and interventions, catering to the unique needs and preferences of individuals.

\item Music \& Emotions: The course explored the complex relationship between music and emotions, highlighting the distinction between perceived and felt emotions. Perceived emotions refer to the emotions that listeners attribute to the music itself, while felt emotions are the emotions experienced by the listeners in response to the music. This distinction is crucial for understanding the emotional impact of music and can inform the design of music-based interventions for emotional regulation and well-being.

\item Relevance of Context: The course underscored the importance of context in shaping music perception and preference. Factors such as the listener's mood, the environment, and the cultural background can greatly influence the emotional responses to music. By considering the relevance of context, we can better understand the variability in music experiences and develop strategies to optimize the emotional impact of music in various settings.

\item Big Music Data: The course introduced the growing field of big music data, which involves the analysis of large-scale music datasets to uncover patterns and trends in music preferences and consumption. The intersection of big music data with social media offers new opportunities for understanding music preferences on a global scale, informing the music industry and providing insights into cultural trends and dynamics.

\item Psychoacoustics \& Audio Signal Processing: The course provided a solid foundation in psychoacoustics and audio signal processing, equipping students with essential tools for analyzing and understanding the acoustic features of music and their impact on perception and cognition. Topics such as audition, pitch, loudness, timbre, and polyphonic timbre were covered, offering a comprehensive understanding of the auditory aspects of music and their role in shaping our musical experiences.

\item Cross-modal Perception: The course delved into the fascinating world of cross-modal perception, exploring the interplay between different sensory modalities and how they can influence our experience of music. This knowledge can be applied to create multisensory experiences that enhance music enjoyment and can inform the design of products and environments that capitalize on the synergistic effects of combining different sensory modalities.

\end{enumerate}

Our team project,``The Beat of Typing: Does Music Make you Type Better?", aligned closely with several aspects of the course, particularly the topics of cross-modal perception, psychoacoustics, and audio signal processing. These concepts were helpful in designing our study, as they allowed us to explore the potential influence of music on typing performance and to analyze the acoustic features of the music used in the experiment.

The theoretical concepts that grabbed my attention the most were digital filters, neuroscience, and the introduction to Fourier Transform. Digital filters provided insights into how we can manipulate and analyze audio signals, while the introduction to Fourier Transform offered a powerful mathematical tool for understanding the frequency components of complex signals. Neuroscience, on the other hand, shed light on the neural mechanisms underlying music perception and cognition, allowing for a deeper understanding of the brain's response to music.

The specific learnings from this course can find applications in various fields beyond music cognition. For instance, digital filters and Fourier Transform techniques can be applied in telecommunications, image processing, and biomedical signal analysis. Neuroscience findings can inform the development of music therapy interventions for various mental health conditions, such as depression and anxiety. Cross-modal perception research can be extended to other sensory modalities, such as taste and smell, and can inform the design of multisensory experiences in fields like marketing, product design, and virtual reality.

In conclusion, the ``Music, Mind \& Technology" course offered a rich and diverse exploration of the many facets of music cognition, perception, and emotion. The course not only provided valuable insights for our team project but also sparked my interest in several theoretical concepts with broad applications across various fields. The knowledge gained from this course will undoubtedly continue to inform and inspire my future endeavours in music cognition and beyond.

\begin{thebibliography}{00}
\bibitem{b1} Lammers, Mark \& Kruger, Mark. (2007). Book review: D.J. LEVITIN, This is Your Brain on Music: The Science of a Human Obsession. New York: Dutton, 2006. 314 pp. ISBN 0--525--94969 (hbk) $24.95; 0--45--228852--5 (pbk) $15.00. Psychology of Music - PSYCHOL MUSIC. 36. 260-262. https://doi.org/10.1177/03057356080360020702.

\bibitem{b2} Spence, Charles. (2021). Sensehacking: Sensehacking: How to use the power of your senses for happier, heathier living.

\bibitem{b3} Crisinel, Anne-Sylvie \& Spence, Charles. (2011). A Fruity Note: Crossmodal associations between odors and musical notes. Chemical senses. 37. 151-8. https://doi.org/10.1093/chemse/bjr085.

\bibitem{b4} Knoeferle, Klemens \& Woods, Andrew \& Käppler, Florian \& Spence, Charles. (2015). That Sounds Sweet: Using Cross-Modal Correspondences to Communicate Gustatory Attributes. Psychology and Marketing. 32. https://doi.org/10.1002/mar.20766. 

\bibitem{b5} Spence, Charles. (2011). Crossmodal correspondences: A tutorial review. Attention, perception \& psychophysics. 73. 971-95. https://doi.org/10.3758/s13414-010-0073-7. 

\bibitem{b6} Deroy, Ophelia \& Michel, Charles \& Piqueras-Fiszman, Betina \& Spence, Charles. (2014). The plating manifesto (I): From decoration to creation. Flavour. 3. https://doi.org/10.1186/2044-7248-3-6. 

\bibitem{b7} Rentfrow, P. J., \& Gosling, S. D. (2003). The do re mi's of everyday life: The structure and personality correlates of music preferences. Journal of Personality and Social Psychology, 84(6), 1236–1256. https://doi.org/10.1037/0022-3514.84.6.1236

\bibitem{b8} Knoeferle, Klemens \& Spence, Charles. (2012). Crossmodal correspondences between sounds and tastes. Psychonomic bulletin \& review. 19. https://doi.org/10.3758/s13423-012-0321-z. 

\bibitem{b9} Crisinel, Anne-Sylvie \& Spence, Charles. (2012). The impact of pleasantness ratings on crossmodal associations between food samples and musical notes. Food Quality and Preference. 24. 136-140. https://doi.org/10.1016/j.foodqual.2011.10.007.

\bibitem{b10} Wang, Qian \& Spence, Charles. (2015). Assessing the Effect of Musical Congruency on Wine Tasting in a Live Performance Setting. i-Perception. 6. https://doi.org/10.1177/2041669515593027.

\end{thebibliography}

\end{document}
